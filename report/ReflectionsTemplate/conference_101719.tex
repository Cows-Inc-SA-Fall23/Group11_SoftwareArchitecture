\documentclass[conference]{IEEEtran}
\IEEEoverridecommandlockouts
% The preceding line is only needed to identify funding in the first footnote. If that is unneeded, please comment it out.
\usepackage{cite}
\usepackage{amsmath,amssymb,amsfonts}
\usepackage{algorithmic}
\usepackage{graphicx}
\usepackage{textcomp}
\usepackage{xcolor}

\usepackage{multirow}
\usepackage{rotating}

\usepackage{mdframed}
\usepackage{hyperref}
\usepackage{tikz}
\usepackage{makecell}
\usepackage{tcolorbox}
\usepackage{amsthm}
%\usepackage[english]{babel}
\usepackage{pifont} % checkmarks
%\theoremstyle{definition}
%\newtheorem{definition}{Definition}[section]


\usepackage{listings}
\lstset
{ 
    basicstyle=\footnotesize,
    numbers=left,
    stepnumber=1,
    xleftmargin=5.0ex,
}


%SCJ
\usepackage{subcaption}
\usepackage{array, multirow}
\usepackage{enumitem}


\def\BibTeX{{\rm B\kern-.05em{\sc i\kern-.025em b}\kern-.08em
    T\kern-.1667em\lower.7ex\hbox{E}\kern-.125emX}}
\begin{document}

%\IEEEpubid{978-1-6654-8356-8/22/\$31.00 ©2022 IEEE}
% @Sune:
% Found this suggestion: https://site.ieee.org/compel2018/ieee-copyright-notice/
% I have added it - you can see if it fulfills the requirements

%\IEEEoverridecommandlockouts
%\IEEEpubid{\makebox[\columnwidth]{978-1-6654-8356-8/22/\$31.00 ©2022 IEEE %\hfill} \hspace{\columnsep}\makebox[\columnwidth]{ }}
%978-1-6654-8356-8/22/$31.00 ©2022 IEEE
% copyright notice added:
%\makeatletter
%\setlength{\footskip}{20pt} 
%\def\ps@IEEEtitlepagestyle{%
%  \def\@oddfoot{\mycopyrightnotice}%
%  \def\@evenfoot{}%
%}
%\def\mycopyrightnotice{%
%  {\footnotesize 978-1-6654-8356-8/22/\$31.00 ©2022 IEEE\hfill}% <--- Change here
%  \gdef\mycopyrightnotice{}% just in case
%}


\title{Reflection Report Template\\
}

\author{
    \IEEEauthorblockN{
        Student 1
    }
    \IEEEauthorblockA{
        University of Southern Denmark, SDU Software Engineering, Odense, Denmark \\
        Email: student1@mmmi.sdu.dk
    }
}


%%%%

%\author{\IEEEauthorblockN{1\textsuperscript{st} Blinded for review}
%\IEEEauthorblockA{\textit{Blinded for review} \\
%\textit{Blinded for review}\\
%Blinded for review \\
%Blinded for review}
%\and
%\IEEEauthorblockN{2\textsuperscript{nd} Blinded for review}
%\IEEEauthorblockA{\textit{Blinded for review} \\
%\textit{Blinded for review}\\
%Blinded for review \\
%Blinded for review}
%\and
%\IEEEauthorblockN{3\textsuperscript{nd} Blinded for review}
%\IEEEauthorblockA{\textit{Blinded for review} \\
%\textit{Blinded for review}\\
%Blinded for review \\
%Blinded for review}
%}

%%%%
%\IEEEauthorblockN{2\textsuperscript{nd} Given Name Surname}
%\IEEEauthorblockA{\textit{dept. name of organization (of Aff.)} \\
%\textit{name of organization (of Aff.)}\\
%City, Country \\
%email address or ORCID}


\maketitle
\IEEEpubidadjcol


\section{Contribution}



\section{Discussion}
\subsection{Achieved Design Goals}
In the proposed Industry 4.0 livestock farming system, the main design goal as stated in the paper is to achieve sufficient interoperability, availability, and deployability of the production software. After thorough review and discussion we concluded that the solution appears to be successful in complying to our quality attributes, supported by the architectural design choices and the use of technologies like Apache Kafka, Java-based subsystems, Docker and Hadoop Distributed File System.

\subsubsection{Interoperability}
The use of Apache Kafka and Java-based subsystems support well the interoperability requirement. The easy implementation in Java-based subsystems, the Java's ability to avoid dependence of existing operating system or hardware and the ability to connect and communicate with a wide variety of systems makes Java-based subsystems ideal interoperability choice. Moreover, Kafka's high compatibility with various systems and programming languages plays a crucial role in ensuring strong and on-going communication between subsystems.

\subsubsection{Availability}
The system's inclusion of Kafka’s fault tolerance through data replication and Java's flexible memory management capabilities supports a high sufficiency of availability. This is critical for both sensor data traffic and storage, in which system reliability is crucial.

\subsubsection{Deployability}
Lack of modification need in Java-based systems and the use of Docker for containerization supports the continuous deployability of the system. Our design succeeds in supporting the requirement of the system to be adaptable to future changes such as increased livestock data without any significant downtime. Furthermore, Docker's ability to provide a consistent environment across different platforms is a vital element in achieving deployability.



\section{Reflection}

Reflection is for noobs


\section{Conclusion}


\bibliographystyle{IEEEtran}
\bibliography{references}
\vspace{12pt}
\end{document}
