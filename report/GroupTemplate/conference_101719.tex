\documentclass[conference]{IEEEtran}
\IEEEoverridecommandlockouts
% The preceding line is only needed to identify funding in the first footnote. If that is unneeded, please comment it out.
\usepackage{cite}
\usepackage{amsmath,amssymb,amsfonts}
\usepackage{algorithmic}
\usepackage{graphicx}
\usepackage{textcomp}
\usepackage{xcolor}

\usepackage{multirow}
\usepackage{rotating}

\usepackage{mdframed}
\usepackage{hyperref}
\usepackage{tikz}
\usepackage{makecell}
\usepackage{tcolorbox}
\usepackage{amsthm}
%\usepackage[english]{babel}
\usepackage{pifont} % checkmarks
%\theoremstyle{definition}
%\newtheorem{definition}{Definition}[section]


\usepackage{listings}
\lstset
{ 
    basicstyle=\footnotesize,
    numbers=left,
    stepnumber=1,
    xleftmargin=5.0ex,
}


%SCJ
\usepackage{subcaption}
\usepackage{array, multirow}
\usepackage{enumitem}


\def\BibTeX{{\rm B\kern-.05em{\sc i\kern-.025em b}\kern-.08em
    T\kern-.1667em\lower.7ex\hbox{E}\kern-.125emX}}
\begin{document}

%\IEEEpubid{978-1-6654-8356-8/22/\$31.00 ©2022 IEEE}
% @Sune:
% Found this suggestion: https://site.ieee.org/compel2018/ieee-copyright-notice/
% I have added it - you can see if it fulfills the requirements

%\IEEEoverridecommandlockouts
%\IEEEpubid{\makebox[\columnwidth]{978-1-6654-8356-8/22/\$31.00 ©2022 IEEE %\hfill} \hspace{\columnsep}\makebox[\columnwidth]{ }}
%978-1-6654-8356-8/22/$31.00 ©2022 IEEE
% copyright notice added:
%\makeatletter
%\setlength{\footskip}{20pt} 
%\def\ps@IEEEtitlepagestyle{%
%  \def\@oddfoot{\mycopyrightnotice}%
%  \def\@evenfoot{}%
%}
%\def\mycopyrightnotice{%
%  {\footnotesize 978-1-6654-8356-8/22/\$31.00 ©2022 IEEE\hfill}% <--- Change here
%  \gdef\mycopyrightnotice{}% just in case
%}


\title{Group Report Template\\
}

\author{
    \IEEEauthorblockN{
        Stefan-Daniel Horvath\IEEEauthorrefmark{1},
        Hi IM Micahel\IEEEauthorrefmark{1},
        Student 3\IEEEauthorrefmark{1},
        Student 4\IEEEauthorrefmark{1},
        Student 5\IEEEauthorrefmark{1},
    }
    \IEEEauthorblockA{
        University of Southern Denmark, SDU Software Engineering, Odense, Denmark \\
        Email: \IEEEauthorrefmark{1} \textnormal{\{Stefan-Daniel Horvath,student2,student3,student4,student5\}}@student.sdu.dk
    }
}


%%%%

%\author{\IEEEauthorblockN{1\textsuperscript{st} Blinded for review}
%\IEEEauthorblockA{\textit{Blinded for review} \\
%\textit{Blinded for review}\\
%Blinded for review \\
%Blinded for review}
%\and
%\IEEEauthorblockN{2\textsuperscript{nd} Blinded for review}
%\IEEEauthorblockA{\textit{Blinded for review} \\
%\textit{Blinded for review}\\
%Blinded for review \\
%Blinded for review}
%\and
%\IEEEauthorblockN{3\textsuperscript{nd} Blinded for review}
%\IEEEauthorblockA{\textit{Blinded for review} \\
%\textit{Blinded for review}\\
%Blinded for review \\
%Blinded for review}
%}

%%%%
%\IEEEauthorblockN{2\textsuperscript{nd} Given Name Surname}
%\IEEEauthorblockA{\textit{dept. name of organization (of Aff.)} \\
%\textit{name of organization (of Aff.)}\\
%City, Country \\
%email address or ORCID}



\maketitle
\IEEEpubidadjcol
\begin{abstract}
    %%%%%%%%%%%%%%%%%% Max 970 signs without space %%%%%%%%%%%%%%%%%%
    % Intro

    % Gab

    % Aim 

    % Method

    % Results 

\end{abstract}

\begin{IEEEkeywords}
    Keyword1, Keyword2, Keyword3, Keyword4, Keyword5
\end{IEEEkeywords}

\section{Introduction and Motivation}
%Introduction and motivate the problem
The advancement of technology is at an all-time high and the possibilities using technology are expanding rapidly. Several industries are benefitting using assembly robots, self-driving forklifts, etc. People, machines, and products are directly related to each other. The term "Industry 4.0" describes \textit{the intelligent networking of machines and processes for industry with the help of information and communication technology} \cite{plattformindustrie40}. The principles of Industry 4.0 result in several benefits, including the possibility of flexibility within production, ability to quickly change a production line for a different task, and many more \cite{plattformindustrie40}. \newline
This concept introduces the need for many different systems to be able to communicate in one way or another, which is one of the big challenges Industry 4.0 Is faced with \cite{PANETTO2019198}. Therefore, for a system to successfully adhere to Industry 4.0 principles it must be capable of communicating between subsystems and various types of hardware within the system. \newline
In this paper the intent of the project is to create a system that can handle various aspects of production, including monitoring biometrics, managing feed, and handling orders in a fully automated livestock farming system. \newline
\textit{Therefore this paper designs and implements an architecture based on a smart livestock farming system. The system is evaluated by its ability to adhere to stated quality attributes}.
The motivation behind designing such a system is based on the need for interoperability between different systems, which in its current state in farming is one of the major challenges. The lack of communication between these technologies hampers their synergistic use by farmers \cite{app122412844}. Through our work, we aim to address these challenges and contribute to the seamless integration of diverse technologies.






The structure of the paper is as follows.
Section \ref{sec:problem} outlines the research question and the research approach.
%to analyze the research question and evaluate our results.
Section \ref{sec:related_work} describes similar work in the field and how our contribution fits the field.
Section \ref{sec:use_case} presents a use case of the system, which gives us a better representation of what the system is able to do.
The use case serves as input to specify QA requirements for the system \ref{sec:qas}.
Section \ref{sec:middleware_architecture} introduces the proposed software architecture design for the system.
Section \ref{sec:evaluation} evaluates the proposed architecture on tests conducted on the system and analyzes the results against the stated QA requirement.

\section{Problem and Approach}

\label{sec:problem}
\emph{Problem.}


\emph{Research questions:}
\begin{enumerate}
    \item
    \item
\end{enumerate}


\emph{Approach.}
The following steps are taken to answer this paper's research questions:
\begin{enumerate}
    \item
\end{enumerate}



\section{Related work}
\label{sec:related_work}
This Section addresses existing contributions by examining xxx in the I4.0 domain.
In total, x papers are investigated.

In \cite{Wan2019Reconfigurable}, experiences are elaborated on a three-layer architecture of a reconfigurable smart factory for drug packing in healthcare I4.0.


The paper \cite{Yazen2010Ontology} proposes an ontology agent-based architecture for inferring  new configurations to adapt to changes in manufacturing requirements and/or environment.



In \cite{Leitao2016Specification,Angione2017Integration} an architecture for a reconfigurable production system is specified.
Two objectives for reconfiguration and how they can be reached are described.


Several papers \cite{Koren1999Reconfigurable,Koren2010Design,Bortolini2018Reconfigurable} describe reconfigurable manufacturing systems that are cost-effective and responsive to market changes.

All contributions provide valuable knowledge about reconfiguration but lack a study of the software architecture perspective that specifies a quantifiable reconfigurability architectural requirement, a software architecture that adopts the architectural requirements, and evaluates the architectural requirement.



\section{Use Case and Quality Attribute Scenario}
\label{sec:use_case_and_qas}
This Section introduces the use case and the specified x QASes.
The QASes are developed based on the use case.

\subsection{Use case}
\label{sec:use_case}





\subsection{Quality attribute scenarios}
\label{sec:qas}










\section{The solution}
\label{sec:middleware_architecture}

% Description of the overall architecture designs
% Argue for tactics used to archieve the QASes
% Discuss the trade-offs

This section will describe a proposed design of that aims to achieve the stated QASes stated in the previous section.






\section{Evaluation}
\label{sec:evaluation}
This Section describes the evaluation of the proposed design.
Section \ref{sec:design} introduces the design of the experiment to evaluate the system.
Section \ref{sec:measurements} identifies the measurements in the system for the experiment.
Section \ref{sec:pilot_test} describes the pilot test used to compute the number of replication in the actual evaluation.
Section \ref{sec:analysis} presents the analysis of the results from the experiment.


\subsection{Experiment design}
\label{sec:design}


\subsection{Measurements}
\label{sec:measurements}


\subsection{Pilot test}
\label{sec:pilot_test}

\subsection{Analysis}
\label{sec:analysis}


\section{Future work}


\section{Conclusion}


\bibliographystyle{IEEEtran}
\bibliography{references}
\vspace{12pt}
\end{document}
