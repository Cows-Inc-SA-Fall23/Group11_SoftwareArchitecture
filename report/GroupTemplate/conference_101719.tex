\documentclass[conference]{IEEEtran}
\IEEEoverridecommandlockouts
% The preceding line is only needed to identify funding in the first footnote. If that is unneeded, please comment it out.
\usepackage{cite}
\usepackage{amsmath,amssymb,amsfonts}
\usepackage{algorithmic}
\usepackage{graphicx}
\usepackage{textcomp}
\usepackage{xcolor}

\usepackage{multirow}
\usepackage{rotating}
\usepackage{tabularx,ragged2e,booktabs}
\usepackage{float}

\usepackage{mdframed}
\usepackage{hyperref}
\usepackage{tikz}
\usepackage{makecell}
\usepackage{tcolorbox}
\usepackage{amsthm}
%\usepackage[english]{babel}
\usepackage{pifont} % checkmarks
%\theoremstyle{definition}
%\newtheorem{definition}{Definition}[section]


\usepackage{listings}
\lstset
{ 
    basicstyle=\footnotesize,
    numbers=left,
    stepnumber=1,
    xleftmargin=5.0ex,
}


%SCJ
\usepackage{subcaption}
\usepackage{array, multirow}
\usepackage{enumitem}


\def\BibTeX{{\rm B\kern-.05em{\sc i\kern-.025em b}\kern-.08em
    T\kern-.1667em\lower.7ex\hbox{E}\kern-.125emX}}
\begin{document}

%\IEEEpubid{978-1-6654-8356-8/22/\$31.00 ©2022 IEEE}
% @Sune:
% Found this suggestion: https://site.ieee.org/compel2018/ieee-copyright-notice/
% I have added it - you can see if it fulfills the requirements

%\IEEEoverridecommandlockouts
%\IEEEpubid{\makebox[\columnwidth]{978-1-6654-8356-8/22/\$31.00 ©2022 IEEE %\hfill} \hspace{\columnsep}\makebox[\columnwidth]{ }}
%978-1-6654-8356-8/22/$31.00 ©2022 IEEE
% copyright notice added:
%\makeatletter
%\setlength{\footskip}{20pt} 
%\def\ps@IEEEtitlepagestyle{%
%  \def\@oddfoot{\mycopyrightnotice}%
%  \def\@evenfoot{}%
%}
%\def\mycopyrightnotice{%
%  {\footnotesize 978-1-6654-8356-8/22/\$31.00 ©2022 IEEE\hfill}% <--- Change here
%  \gdef\mycopyrightnotice{}% just in case
%}


\title{Group Report Template\\
}

\author{
    \IEEEauthorblockN{
        Stefan-Daniel Horvath\IEEEauthorrefmark{1},
        Hi IM Micahel\IEEEauthorrefmark{1},
        Student 3\IEEEauthorrefmark{1},
        Student 4\IEEEauthorrefmark{1},
        Student 5\IEEEauthorrefmark{1},
    }
    \IEEEauthorblockA{
        University of Southern Denmark, SDU Software Engineering, Odense, Denmark \\
        Email: \IEEEauthorrefmark{1} \textnormal{\{Stefan-Daniel Horvath,student2,student3,student4,student5\}}@student.sdu.dk
    }
}


%%%%

%\author{\IEEEauthorblockN{1\textsuperscript{st} Blinded for review}
%\IEEEauthorblockA{\textit{Blinded for review} \\
%\textit{Blinded for review}\\
%Blinded for review \\
%Blinded for review}
%\and
%\IEEEauthorblockN{2\textsuperscript{nd} Blinded for review}
%\IEEEauthorblockA{\textit{Blinded for review} \\
%\textit{Blinded for review}\\
%Blinded for review \\
%Blinded for review}
%\and
%\IEEEauthorblockN{3\textsuperscript{nd} Blinded for review}
%\IEEEauthorblockA{\textit{Blinded for review} \\
%\textit{Blinded for review}\\
%Blinded for review \\
%Blinded for review}
%}

%%%%
%\IEEEauthorblockN{2\textsuperscript{nd} Given Name Surname}
%\IEEEauthorblockA{\textit{dept. name of organization (of Aff.)} \\
%\textit{name of organization (of Aff.)}\\
%City, Country \\
%email address or ORCID}



\maketitle
\IEEEpubidadjcol
\begin{abstract}
    %%%%%%%%%%%%%%%%%% Max 970 signs without space %%%%%%%%%%%%%%%%%%
    % Intro

    % Gab

    % Aim 

    % Method

    % Results 

\end{abstract}

\begin{IEEEkeywords}
    Keyword1, Keyword2, Keyword3, Keyword4, Keyword5
\end{IEEEkeywords}

\section{Introduction and Motivation}
%Introduction and motivate the problem
The advancement of technology is at an all-time high and the possibilities using technology are expanding rapidly. Several industries are benefitting using assembly robots, self-driving forklifts, etc. People, machines, and products are directly related to each other. The term "Industry 4.0" describes \textit{the intelligent networking of machines and processes for industry with the help of information and communication technology} \cite{plattformindustrie40}. The principles of Industry 4.0 result in several benefits, including the possibility of flexibility within production, ability to quickly change a production line for a different task, and many more \cite{plattformindustrie40}. \vspace{2mm} \newline
This concept introduces the need for many different systems to be able to communicate in one way or another, which is one of the big challenges Industry 4.0 is faced with \cite{PANETTO2019198}. Therefore, for a system to successfully adhere to Industry 4.0 principles it must be capable of communicating between subsystems and various types of hardware within the system. \newline
Industry 4.0 might originally have been coind for the manufacturing industry, but the potential of the concept is not limited to this industry alone. There is great potential for farming to adapt to the principles also. By incorporating the power of real-time data, IoT (internet of things) and analytics, farmers can respond to more nuanced changes in the environnment and the health of the animals. This not only has great benefits for the productivity of the farm, but also has benefits for a more sustainable farming industry. \newline
In this paper the intent of the project is to create a system that can handle various aspects of production, including monitoring biometrics, managing feed, and handling orders in a fully automated livestock farming system. \vspace{2mm} \newline
\textit{Therefore this paper designs and implements an architecture based on a smart livestock farming system. The system is evaluated by its ability to adhere to stated quality attributes}.
The motivation behind designing such a system is based on the need for interoperability between different systems, which in its current state in farming is one of the major challenges. The lack of communication between these technologies hampers their synergistic use by farmers \cite{app122412844}. Through our work, we aim to address these challenges and contribute to the seamless integration of diverse technologies. \vspace{3mm}






The structure of the paper is as follows.
Section \ref{sec:problem} outlines the research question and the research approach.
%to analyze the research question and evaluate our results.
Section \ref{sec:related_work} describes similar work in the field and how our contribution fits the field.
Section \ref{sec:use_case} presents a use case of the system, which gives us a better representation of what the system is able to do.
The use case serves as input to specify QA requirements for the system \ref{sec:qas}.
Section \ref{sec:middleware_architecture} introduces the proposed software architecture design for the system.
Section \ref{sec:evaluation} evaluates the proposed architecture on tests conducted on the system and analyzes the results against the stated QA requirement.

\vspace{1cm}

\section{Problem and Approach}
% conext:
% This is a secitoin talking about the problem that we are facing in this ?project. The project is about Indystry 4.0 in a livestock farm setting.
% we have differnet services and serves in a Java-based agricultural management system for cattle farming that integrates several services: an Outbound Department for order management, a Laboratory/QA for product testing and compliance monitoring, and a Farmer GUI for user interactions. It employs a Message Bus for inter-service communication and Bio-monitoring for livestock health tracking, utilizing both MySQL and semi-structured databases. An Inbound service manages stock and orders, while the Feed System oversees feed conditions and inventory. The system is sensor-reliant, with various sensors providing real-time data on cattle health, environmental conditions, and feed storage, interfacing with a Hadoop big data server for analytics and reporting.

\label{sec:problem}
\emph{Problem.}
In livestock farming, the production process is highly dependent on the environment and the health of the animals. To address this the system that has been developed in this project aims to look into how different architectural approaches can bridge different services and servers in a way that the production process can be reconfigured to adapt to the changes in the environment and the health of the animals. The problem of catching decese ealy and also trying to automating asspetcts of the farming production line could lead to more optimized environment both for the livestock but also for the farmer. It is belived that by devloping such a interoperable system it would lead to better healf for the animals and more time for the farmer to focus on other things where needed.To approach this problem the following research questions are formulated: \vspace{1mm}


\emph{Research questions:}
\begin{enumerate}
    \item How can different architectures support the stated production system requirements?
    \item Which architectural tradeoffs must be taken due to the technology choices?
\end{enumerate}


\emph{Approach.}
The following steps are taken to answer this paper's research questions:
\begin{enumerate}
    \item Develop the overall architecture of the system.
          \begin{enumerate}
              \item Identify the services and servers that are needed to support the production system.
              \item Identify the usecases that the system should support.
              \item Identify the Quality Attributes that the system should support and how they are prioritized.
              \item Identify the non-functional requirements that the system should support.
          \end{enumerate}
    \item Research the technologies that could be used to support the system and the tradeoffs that are made by using them.
    \item Develop a prototype of the system.
    \item Evaluate the prototype based on the Quality Attributes that are identified in step 1.
    \item Analyze the results and answer the research questions.
\end{enumerate}

\vspace{1cm}

\section{Related work}
\label{sec:related_work}
This section addresses existing knowledge and contributions by examining Industry 4.0 and its current use within the livestock and agricultural domain. In total, eight papers are investigated using a systematic literature review. \vspace{2mm}
\newline
Over the past years, large parts of the industry have been undergoing the revolution to Industry 4.0, integrating higher levels of technology into the lifecycle of products. Order systems, production environments and shipment as well as maintenance of products became more automated and integrated different, interconnected smart systems.
However, the primary sector, mainly the agricultural part of industry, has not kept up with the rapid changes to modern production standards \cite{pr7010036}.
This project plan to design an Industry 4.0 architecture for a prototype system for the livestock farming domain. For that, the project begins with evaluating which needs and challenges are already reported on for similar systems in the domain and find existing suggestions for designing such systems.
In detail, a small but concise systematic literature review is performed considering the following research questions:
\vspace{2mm}
\begin{itemize}
    \item What functional requirements are considered important for agricultural industry 4.0 systems in existing literature?
    \item Do suggestions for parts of the architecture and technologies of systems exist that support the requirements of agricultural industry 4.0 systems?
    \item What are current challenges within agricultural industry 4.0 systems in use?
\end{itemize}
\vspace{2mm}

To answer these questions, an examination of the scientific reports ragarding Industry 4.0 concepts and software in the livestock farming and agricultural context is conducted. For the process of identifying appropriate research, a semi-automated search strategy as well as snowball techniques after finding promising starting point papers. This search is conducted across a spectrum of academic digital databases, including MPBI, ProQuest, DOAJ, Google Scholar, and ScienceDirect. A set of inclusion criteria was utilized with the purpose of gathering appropriate articles, reviews and case studies published in English within a timeline of the last 7 years. This approach is designed to collate a comprehensive and current body of knowledge, providing a solid foundation for the study.
\vspace{2mm}
\\
The keyword search focused on a list of keywords and key phrases related to the research questions. That list included the words Industry 4.0, agriculture, agriculture 4.0, farming, livestock, and any related phrases. Evaluating the search results, the ones which were applicable to the above-mentioned criteria and related to the research questions and challenges were chosen (excluding some papers with a focus on machine learning, since they were not related to the architectural parts of software design).
Additionally, ensuring that the papers present a balanced view by considering various perspectives and supporting their arguments with evidence from diverse sources, such as references and/or observed data was important. It was also checked that the authors have taken care to present the information objectively, avoiding bias in their analysis.
\vspace{2mm}
\newline
To report similar or different aspects covered in the different papers, it was determined three categories covered in most of the papers. These categories are “Functional requirements and quality attributes”, “Architectural aspects” and “Existing challenges/problems”. For each of them, an overview of the most important points made is given below.
\newline
\subsection{Functional requirements and quality attributes}
The most common quality attribute for agriculture 4.0 systems throughout the list of papers is interoperability of different subsystems. All the listed papers either mention this attribute directly or refer to middleware products, which are the heart of interoperability between different subsystems.
Similarly, flexibility of the system, for example being able to add new subsystems or change parts of the software, is addressed as an important attribute in several papers \cite{KRUIZE201612}, \cite{app122412844}, \cite{jepsen2021analysis}, \cite{10092655}.
Another common attribute is scalability, which papers \cite{KRUIZE201612}, \cite{agronomy12030750}, \cite{s22124319} directly address. In the context of a farm there is a possibility of it growing over time, it is therefore important that the system capable of scaling and can adapt to larger amounts of data and the added workload.
A few of the papers \cite{agronomy12030750}, \cite{s22124319} also mentions security as an important attribute of large-scale systems, so that the data and system can't be tampered with.
While there were little contradicting opinions on attributes, different papers had different focus points. A few attributes worth mentioning include component availability \cite{KRUIZE201612}, reconfigurability \cite{10092655} or accuracy of systems \cite{ZHANG2021127712}.
\newline
\subsection{Architectural aspects}
Several of the papers formulate guidelines or opinions on different architectural aspects of agricultural 4.0 systems. To address the above mentioned interoperability requirement, most papers \cite{KRUIZE201612}, \cite{app122412844}, \cite{agronomy12030750}, \cite{jepsen2021analysis}, \cite{10092655}, \cite{ZHANG2021127712} directly focus on middleware components that are used to orchestrate and abstract between different subsystems. In this context, they also mention well-defined interfaces for the communication between different components.
Another directly addressed aspect is the virtualization \cite{pr7010036}, \cite{KRUIZE201612} of different components in the actual physical farm, so that they can be included in the software ecosystem.
While different explicit technologies are addressed by different papers (Docker, Kafka, etc.), a more abstract technological term across different papers is “IoT” \cite{pr7010036}, \cite{agronomy12030750}, \cite{s22124319}, \cite{ZHANG2021127712}. Since many systems in the agricultural context rely on sensor data and physical devices (robots, vehicles), the inclusion of IoT devices is necessary for agriculture 4.0 systems.
\newline
\subsection{Existing challenges/problems}
Several of the papers \cite{pr7010036}, \cite{KRUIZE201612}, \cite{app122412844}, \cite{agronomy12030750}, \cite{s22124319}, \cite{jepsen2021analysis}, \cite{ZHANG2021127712} come to the conclusion that many technologies exist that try to make livestock farming smarter and more efficient, but the problem with the technologies is that often coordination problems occur making the interoperability difficult to achieve. A gap in this is the need for common communication protocols or orchestration middleware that these technologies can share so data can be used for shared interests.
Additional reported challenges exist within the integration of 4.0 systems to the actual farms, which include high financial costs and missing expertise for technological equipment and systems \cite{app122412844}, \cite{s22124319}, \cite{ZHANG2021127712}. To counteract some of these issues, there could be governmental policies, which are at this point still undeveloped \cite{pr7010036}.
Another reported issue might be bandwidth problems in rural areas, which would restrict large interconnected systems with many components \cite{pr7010036}, \cite{s22124319}.
\newline

All contributions provide valuable knowledge about Industry 4.0 and its usage within modern agriculture and livestock farming. However, a notable gap in existing literature is the absence of real-world examples of successful or unsuccessful implementations of livestock farming systems that apply Industry 4.0 principles. Developing practical examples can help provide a clearer understanding of how to solve the challenges of Industry 4.0 within the livestock farming domain.

\section{Use Case and Quality Attribute Scenario}
\label{sec:use_case_and_qas}
This Section introduces the use cases and the specified QASes.
The QASes are developed based on the use cases.



\subsection{Use cases}
\label{sec:use_case}
To better grasp the ability of the system, several use cases have been constructed. The use cases assists to identify systems and subsystems needed to implement the system.
We must also consider the quality of the system that goes beyond just functionality. This will be refered to as quality attributes \cite{bass2021software}.

\subsubsection{Automatic feed ordering}
The first use case outlines an automatic order placement for more feed, which is triggered by low feed stock levels. This use case contains three actors: A feed distributor, a control system and a feed vendor.
Events in this use case is the feed distributor continously monitors feed levels and when the stock reaches below a certain amount it triggers a signal which gets sent to the control system.
The control system recieves the signal and creates the order for more feed, which it then forwards to the vendor system.
The vendor recieves order and checks for availability of feed for the incoming order. In case of resource shortage the vendor notifies the control system, so that the farmer can take action accordingly.
If resource are available the vendor schedules packaging of the feed and makes sure it is shipped to the farm and the control panel is notified of the order completion. From the time that the feed distributor recieves a signal of low stock the whole sequence must be finished within a resonable timeframe. \newline
From this use case it is possible to derive several systems already for the management of livestock feed. First off for the feed distributor to know how much feed is currently in the silo on the farm, it needs information on the current stock. We imagine that this is a weight sensor in the silo. The feed distributor is then designed to act upon the stock of the silo reaches below a certain point. \newline
The use case also describes a control system which handles the signal from the feed distributor and the creation of the order for the vendor, the vendor in this case is an external system that we do not include in our system.

\subsubsection{Detect changes in livestock conditions}
This use case describes a scenario that can alert the farmer of the wellbeings of the livestock. This use case contains four actors: Sensors, a health analyzer, a control system and the farmer.
Eevnts in this use case is first of initiated by continously data being generated by sensors that are either attached to the livestock or its surroundings. This data is being transmitted into a storage system which then can be used to analyze the health of the livestock.
The health analyzer will then analyze the sensor data and if the data suggests e.g. a disease or other forms of the livestocks condition being worsened the system will transmit a warning to the control panel.
The control panel can be accessed by the farmer and the alert from the health analyzer will be able to tell the farmer which actions are required. The system should be able to handle the addition of new livestock, this will result in more sensors being added which should be completed with no downtime of the system.
In case of added load by the addition of more sensors, the system should be able to handle the increased load by dynamically scaling the storage. \newline
From this use case we can identify more systems to be implemented in the system. The control panel is already mentioned from the previous use case.
A system for the sensors are needed to transmit the data in to some form of storage system. The sensors will be generating a great amount of data, so the system should be able to handle a high throughput of data and also store a large amount of it.
This storage system can then be accessed by the health analyzer, which should be able to able to alert the farmer and also inform of the necessary actions. \newline

\subsection{Quality attribute scenarios}
\label{sec:qas}
To define success criteria for the system, several quality attribute scenarios have been created. These are used for setting up a scenario in which one of the quality attributes of the system gets addressed \cite{bass2021software}. This is helpful for when doing testing on the system later.
A template for defining QASes is used for the creation of these scenarios. The templates captures a scenario through \textit{Source, Stimulus, Artifact, Environment, Response and Response measure} to which the requirement should apply \cite{bass2021software}. \newline
For the system in this paper there is specified three main requirements:
\begin{enumerate}
    \item Production software must be able to exchange and coordinate information to execut a prodcution and change production.
    \item Production software must run 24/7
    \item Production software must be continously deployable
\end{enumerate}
These three requirements can be directly translated to quality attributes. The first one refers to interoperability, the second availability and the third is described by deployability. \newline

\subsubsection{Interoperability}

\begin{table}[h]
    \renewcommand{\arraystretch}{1.3}
    \caption{Interoperability quality attribute scenarios}
    \label{interoperability}
    \centering
    \begin{tabularx}{\columnwidth}{>{\hsize=0.3\hsize}X>{\hsize=0.7\hsize}X}
        \hline
        \textbf{Portion of scenario} & \textbf{Value}                                                                               \\
        \hline
        Source                       & Feed distributor system.                                                                     \\
        Stimulus                     & A signal from the feed distributor indicating low feed stock.                                \\
        Artifact                     & Control system.                                                                              \\
        Environment                  & Normal operation during low feed stock.                                                      \\
        Response                     & The control system successfully creates and forwards an order to the external vendor system. \\
        Response measure             & The order is accuratly transmitted and the external vendor acknowledges it wihtin 5 seconds. \\
        \hline
    \end{tabularx}
\end{table}

Table \ref{interoperability} is a scenario for the interoperability quality attribute, which referes to the ability of two or more systems to be able to exchange and use information \cite{brownsword_2004}.
For the interoperability QAS, the scenario described is the ability of the system to be able to react to low feed stock and therefore make sure the necessary communication is carried out with other systems, this includes both the external vendor but also the control system.
The \textit{Source} in this scenario is the feed distributor system, which triggers the \textit{Stimulus}, in this case the indication of low feed stock. The stimulus arrives at the \textit{Artifact} (the control system), which is responsible for creating the order for feed. The \textit{Environment} assumes the condition of the component / system, which in this case is assumed to be under normal operation.
The \textit{Response} describes a process that handles the stimulus, so the system should be able to successfully create and forward the order. To be able to determine if the requirement from the scenario is satisfied we specify the \textit{Response measure}, which in this scenario specifies that the whole process should be transmitted and acknowledged within a certain timeframe (in this case 5 seconds). \newline

\subsubsection{Availability}

\begin{table}[h]
    \renewcommand{\arraystretch}{1.3}
    \caption{Availability quality attribute scenarios}
    \label{availability}
    \centering
    \begin{tabularx}{\columnwidth}{>{\hsize=0.3\hsize}X>{\hsize=0.7\hsize}X}
        \hline
        \textbf{Portion of scenario} & \textbf{Value}                                                                                           \\
        \hline
        Source                       & High sensor data throughput.                                                                             \\
        Stimulus                     & The message bus experiences an overload, causing the loss of a partition.                                \\
        Artifact                     & Message bus.                                                                                             \\
        Environment                  & Overloaded condition.                                                                                    \\
        Response                     & Due to the replication factor, the system ensures that the lost partition's data is still available.     \\
        Response measure             & The system maintains data integrity and availability through replication, even when a partition is lost. \\
        \hline
    \end{tabularx}
\end{table}

Table \ref{availability} is a scenario for the availability quality attribute, which refers to the systems ability to be ready to carry out a task, but it also covers the systems ability to mask or repair faults \cite{bass2021software}.
For the availability QAS, the scenario described is the systems ability to handle added load from the throughput of sensor data. The system should from the added load be able to circumvent the loss of data or the system becoming degraded in performance.
The \textit{Source} is coming from the sensor data throughput, which as described by the \textit{Stimulus} causes loss of a partition within the \textit{Artifact} (the message bus). The \textit{Environment} is assumed to be in an overloaded condition due to the amount of data causing the loss of a partition in the message bus.
The \textit{Response} handles the stimulus by applyin tactics or patterns to address the availability of the system in the case of a partition loss. The \textit{Response measure} indicates the handling of the fault should result in no data lost and the system to be with no downtime. \newline
\subsubsection{Deployability}

\begin{table}[h]
    \renewcommand{\arraystretch}{1.3}
    \caption{Deployability quality attribute scenarios}
    \label{deployability}
    \centering
    \begin{tabularx}{\columnwidth}{>{\hsize=0.3\hsize}X>{\hsize=0.7\hsize}X}
        \hline
        \textbf{Portion of scenario} & \textbf{Value}                                                                                                        \\
        \hline
        Source                       & Increased livestock population.                                                                                       \\
        Stimulus                     & Addition of more sensors to the livestock monitoring system.                                                          \\
        Artifact                     & Sensor system.                                                                                                        \\
        Environment                  & Deployment of additional sensors.                                                                                     \\
        Response                     & The system allows plug-and-play functionality, allowing quick and straight forward integration of additional sensors. \\
        Response measure             & New sensors can be deployed and be operational without downtime or configuration challenges.                          \\
        \hline
    \end{tabularx}
\end{table}


Table \ref{deployability} is a scenario for the deployability of the system, which refers to the ability to continously update a system without the need for downtime \cite{bass2021software}. The idea is that we want to be able to develop and push out updates to our system as fast as possible.
For the deployability QAS, the scenario described highlights the systems ability to increase the amount sensors in the system. The system should be able to handle the added sensors without the need for downtime in the system. 
The \textit{Source} is the increased livestock population which triggers the \textit{Stimulus} which is the addition of sensors. This is handled within the \textit{Artifact} (the sensor system), and the \textit{Environment} is the deployment of the additional sensors. 
The \textit{Response} describes the systmems ability to allow plug-and-play functionality, that allows quick and easy integration of additional sensors. The \textit{Response measure} implies the process should be completed with no system downtime and configuration challenges. \newline 

\section{The solution}
\label{sec:middleware_architecture}

% Description of the overall architecture designs
% Argue for tactics used to archieve the QASes
% Discuss the trade-offs

This section will describe a proposed design of that aims to achieve the stated QASes stated in the previous section.






\section{Evaluation}
\label{sec:evaluation}
This Section describes the evaluation of the proposed design.
Section \ref{sec:design} introduces the design of the experiment to evaluate the system.
Section \ref{sec:measurements} identifies the measurements in the system for the experiment.
Section \ref{sec:pilot_test} describes the pilot test used to compute the number of replication in the actual evaluation.
Section \ref{sec:analysis} presents the analysis of the results from the experiment.


\subsection{Experiment design}
\label{sec:design}


\subsection{Measurements}
\label{sec:measurements}


\subsection{Pilot test}
\label{sec:pilot_test}

\subsection{Analysis}
\label{sec:analysis}


\section{Future work}


\section{Conclusion}


\bibliographystyle{IEEEtran}
\bibliography{references}
\vspace{12pt}
\end{document}
