\section{Introduction and Motivation}
%Introduction and motivate the problem
The advancement of technology is at an all-time high and the possibilities using technology are expanding rapidly. Several industries are benefitting using assembly robots, self-driving forklifts, etc. People, machines, and products are directly related to each other. The term "Industry 4.0" describes \textit{the intelligent networking of machines and processes for industry with the help of information and communication technology} \cite{plattformindustrie40}. The principles of Industry 4.0 result in several benefits, including the possibility of flexibility within production, ability to quickly change a production line for a different task, and many more \cite{plattformindustrie40}. \vspace{2mm} \newline
This concept introduces the need for many different systems to be able to communicate in one way or another, which is one of the big challenges Industry 4.0 is faced with \cite{PANETTO2019198}. Therefore, for a system to successfully adhere to Industry 4.0 principles it must be capable of communicating between subsystems and various types of hardware within the system. \newline
Industry 4.0 might originally have been coind for the manufacturing industry, but the potential of the concept is not limited to this industry alone. There is great potential for farming to adapt to the principles also. By incorporating the power of real-time data, IoT (internet of things) and analytics, farmers can respond to more nuanced changes in the environnment and the health of the animals. This not only has great benefits for the productivity of the farm, but also has benefits for a more sustainable farming industry. \newline
In this paper the intent of the project is to create a system that can handle various aspects of production, including monitoring biometrics, managing feed, and handling orders in a fully automated livestock farming system. \vspace{2mm} \newline
\textit{Therefore this paper designs and implements an architecture based on a smart livestock farming system. The system is evaluated by its ability to adhere to stated quality attributes}.
The motivation behind designing such a system is based on the need for interoperability between different systems, which in its current state in farming is one of the major challenges. The lack of communication between these technologies hampers their synergistic use by farmers \cite{app122412844}. Through our work, we aim to address these challenges and contribute to the seamless integration of diverse technologies. \vspace{3mm}






The structure of the paper is as follows.
Section \ref{sec:problem} outlines the research question and the research approach.
%to analyze the research question and evaluate our results.
Section \ref{sec:related_work} describes similar work in the field and how our contribution fits the field.
Section \ref{sec:use_case} presents a use case of the system, which gives us a better representation of what the system is able to do.
The use case serves as input to specify QA requirements for the system \ref{sec:qas}.
Section \ref{sec:middleware_architecture} introduces the proposed software architecture design for the system.
Section \ref{sec:evaluation} evaluates the proposed architecture on tests conducted on the system and analyzes the results against the stated QA requirement.