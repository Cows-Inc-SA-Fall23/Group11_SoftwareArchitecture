\section{Problem and Approach}
% conext:
% This is a secitoin talking about the problem that we are facing in this ?project. The project is about Indystry 4.0 in a livestock farm setting.
% we have differnet services and serves in a Java-based agricultural management system for cattle farming that integrates several services: an Outbound Department for order management, a Laboratory/QA for product testing and compliance monitoring, and a Farmer GUI for user interactions. It employs a Message Bus for inter-service communication and Bio-monitoring for livestock health tracking, utilizing both MySQL and semi-structured databases. An Inbound service manages stock and orders, while the Feed System oversees feed conditions and inventory. The system is sensor-reliant, with various sensors providing real-time data on cattle health, environmental conditions, and feed storage, interfacing with a Hadoop big data server for analytics and reporting.

\label{sec:problem}
\emph{Problem.}
In livestock farming, the production process is highly dependent on the environment and the health of the animals. To address this the system that has been developed in this project aims to look into how different architectural approaches can bridge different services and servers in a way that the production process can be reconfigured to adapt to the changes in the environment and the health of the animals. The problem of catching decese ealy and also trying to automating asspetcts of the farming production line could lead to more optimized environment both for the livestock but also for the farmer. It is belived that by devloping such a interoperable system it would lead to better healf for the animals and more time for the farmer to focus on other things where needed.To approach this problem the following research questions are formulated: \vspace{1mm}


\emph{Research questions:}
\begin{enumerate}
    \item How can different architectures support the stated production system requirements?
    \item Which architectural tradeoffs must be taken due to the technology choices?
\end{enumerate}


\emph{Approach.}
The following steps are taken to answer this paper's research questions:
\begin{enumerate}
    \item Develop the overall architecture of the system.
          \begin{enumerate}
              \item Identify the services and servers that are needed to support the production system.
              \item Identify the usecases that the system should support.
              \item Identify the Quality Attributes that the system should support and how they are prioritized.
              \item Identify the non-functional requirements that the system should support.
          \end{enumerate}
    \item Research the technologies that could be used to support the system and the tradeoffs that are made by using them.
    \item Develop a prototype of the system.
    \item Evaluate the prototype based on the Quality Attributes that are identified in step 1.
    \item Analyze the results and answer the research questions.
\end{enumerate}