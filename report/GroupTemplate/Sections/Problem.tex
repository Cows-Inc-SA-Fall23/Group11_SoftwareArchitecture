\section{Problem and Approach}
% conext:
% This is a secitoin talking about the problem that we are facing in this ?project. The project is about Indystry 4.0 in a livestock farm setting.
% we have differnet services and serves in a Java-based agricultural management system for cattle farming that integrates several services: an Outbound Department for order management, a Laboratory/QA for product testing and compliance monitoring, and a Farmer GUI for user interactions. It employs a Message Bus for inter-service communication and Bio-monitoring for livestock health tracking, utilizing both MySQL and semi-structured databases. An Inbound service manages stock and orders, while the Feed System oversees feed conditions and inventory. The system is sensor-reliant, with various sensors providing real-time data on cattle health, environmental conditions, and feed storage, interfacing with a Hadoop big data server for analytics and reporting.

\label{sec:problem}
\emph{Problem.}
Productivity in livestock farming har big implications on the environment. While traditional practices have served the industry, their capacity to adapt to rapid changes remian limited. The Industry 4.0 framework offers a solution through its emphasis on intelligent networking and interoperability across systems. By adapting such a framework, efficiency in farming operations can be significantly improved, leading to higher productivity and potentially higher living environment for livestock. Moreover, this could also benifit the principles of sustainable farming. By addressing the challenges of system integration and real-time response to environmental and health variables, the industry could unlock the potential for a holistic, automated livestock farming model that is scalable and future-ready. \vspace{1mm} \newline


% In livestock farming productivity is highly dependent on the environment and the health of the animals. Traditional farming methods are not able to adapt swiftly to changes within these parameters. Industry 4.0 provides a framework for how to handle these changes in a more efficient way. interoperability between different systems is a common problem and is also the focus of industry 4.0 in general. By developing a system that can handle these changes in a more efficient way could lead to a substantial increase in productivity while also a better environment for the animals. It would also have the added benefits of more sustainable farming practices. \vspace{1mm} \newline

\emph{Research questions:}
\vspace{1mm}
\begin{enumerate}
    \item How can different architectures support the stated production system requirements?
    \item Which architectural tradeoffs must be taken due to the technology choices?
\end{enumerate}
\vspace{4mm}

\emph{Approach.}
The following steps are taken to answer this paper's research questions:
\vspace{1mm}
\begin{enumerate}
    \item Develop the overall architecture of the system.
          \begin{enumerate}
              \item Identify the services and servers that are needed to support the production system.
              \item Identify the usecases that the system should support.
              \item Identify the Quality Attributes that the system should support and how they are prioritized.
              \item Identify the non-functional requirements that the system should support.
          \end{enumerate}
    \item Research the technologies that could be used to support the system and the tradeoffs that are made by using them.
    \item Develop a prototype of the system.
    \item Evaluate the prototype based on the Quality Attributes that are identified in step 1.
    \item Analyze the results and answer the research questions.
\end{enumerate}

